\input logicdocfmt

\heading
User-friendly plain \TeX\ macros for formal logic
Dustin Tucker
\resetat\email{dustin.tucker@colostate.edu}
\strut
Project repository: \url{https://github.com/dusttuck/logic}
Source for this document: \url{https://www.overleaf.com/read/qnnwbwfmfmkz}
\endheading

\sec{Overview}

{\tt logic.tex} is a collection of mutually-independent plain \TeX\ macros for intuitive typesetting of standard formal logic with code that is easy to produce and easy to parse, even for people with no \TeX\ or programming experience.

Here's a list of what the macros cover, with examples on subsequent pages. Click the Overleaf link at the top to view or edit the source for this entire file.

\unorderedlist
\li{\bf Single-character abbreviations for symbols.} For instance,\logicstrut{0pt}{9.5pt}
\linebreak\strut\hfil@Ax[F(x) > -G(x,a)]@\qquad produces\qquad{$Ax[F(x) > -G(x,a)]$}.
\li{\bf Fitch-style proofs}
\li{\bf Truth tables}
\li{\bf Arguments.} Premises and a conclusion separated by a horizontal line.
\li{\bf Diagrams of relative height.} These can be used as countermodels to prove invalidity.
\li@\itmath@{\bf.} Italic math letters with word-like kerning. For instance,\logicstrut{0pt}{9.5pt}
\linebreak\strut\hfil{\itmath$-Sees(ruth,alonzo)$}\qquad instead of\qquad$-Sees(ruth,alonzo)$.
\li@\ssmath@{\bf.} Sans-serif math letters. For instance,\qquad{\ssmath$P^Q$}\qquad instead of\qquad$P^Q$.
\endunorderedlist

(The additional macros used to format this document are included in @logicdocfmt.tex@, some of which depend on @eplain.tex@, but the core macros from @logic.tex@ are entirely independent of those as well.)

I wrote these primarily so that students could complete problem sets, quizzes, and exams on Overleaf. Here are a few examples:

\unorderedlist
\li Translations: \url{https://www.overleaf.com/read/tjmqfmjkqvkh}
\li Truth tables: \url{https://www.overleaf.com/read/yswxpbykzgtj}
\li Fitch-style proofs: \url{https://www.overleaf.com/read/gmbhbqzrbjst}
\endunorderedlist

\pagebreak
\itmath

\sec{Abbreviations}

Three macros make the following characters behave like the listed expressions:

\noindent\hfil\vbox\bgroup\halign{
	\hfil#&\quad\hfil#\hfil&\qquad#\hfil&#\hfil\cr
	              &\bf Character&\bf Expression\cr
	\noalign{\medskip}
	@\connectives@&@-@&@\lnot@&(in math mode)\cr
	              &@v@&@\mathrel\lor@\cr
	              &@^@&@\mathrel\land@\cr
	              &@>@&@\rightarrow@\cr
	              &@<@&@\leftrightarrow@\cr
	              &@!@&@\bot@\cr
	              &@=@&@\mathbin=@\cr
	\noalign{\medskip}
	@\quantifiers@&@A@&@\forall@&(in math mode)\cr
	              &@E@&@\exists@\cr
	\noalign{\medskip}
	@\makeasteriskbig@&@*@&@\big@&(everywhere)\cr
}\egroup\hfil

\noindent They can be ended with @\endconnectives@, @\endquantifiers@, and @\resetasterisk@.

To use $\sim$ and \&\ (abbreviated by @~@ and @&@) or $\supset$ and $\equiv$ (still abbreviated by @>@ and @<@), simply uncomment the relevant lines in the definition of @\connectives@.

\sec{Fitch-style proofs}

@+@ produces additional vertical space between lines. I find that it looks better to use two @+@s between consecutive subproofs:

\fitchproof
 1.| P v Q
 2.| P > R
 3.| Q > R  ..  Goal: R
   |---
 4.|   | P  ..  Assumption  .  Setting up: v Elim  .  Goal: R
   |   |---
 5.|   | R  .  > Elim: 2, 4
   |   +
   |   +
 6.|   | Q  ..  Assumption  .  Setting up: v Elim  .  Goal: R
   |   |---
 7.|   | R  .  > Elim: 3, 6
   |   +
 8.| R      .  v Elim: 1, 4-5, 6-7
\endfitchproof

Proofs (and all the other constructions in @logic.tex@) can be used inline, and all the spacing and bar widths are customizable. To customize the indentation of the idiosyncratic notes I use in my class (goal, setting up, etc.), use @\fitchproofindentby@:

\fitchrowheight=10.5pt \fitchrowdepth=3.5pt%   Rows containing formulas
\fitchbarrowheight=4pt \fitchbarrowdepth=4pt%  Rows containing horizontal bars
\fitchbarlength=12pt%                          Length of horizontal bars
\fitchlinethickness=.8pt%                      Thickness of all lines
\fitchpaddingthickness=2pt%                    Padding produced by +
\fitchspace=4pt%                               Governs all horizontal spacing

{\bf Example. }
\fitchproofindentby{-12pt}
  1. | Ex Ay F(x,y)
  2. | Ax*[Ey F(y,x) > G(x)*]    ..    Goal: Ax G(x)
     |---
     |   +
  3. |   | :a: Ay F(a,y)         ..    Assumption . Setting up: E Elim   .  Goal: Ax G(x)
     |   |---
     |   |   +
  4. |   |   | :b:               ..    Assumption . Setting up: A Intro  .  Goal: G(b)
     |   |   |---
  5. |   |   | Ey F(y,b) > G(b)  .  A Elim: 2
  6. |   |   | F(a,b)            .  A Elim: 3
  7. |   |   | Ey F(y,b)         .  E Intro: 6
  8. |   |   | G(b)              .  > Elim: 5, 7
     |   |   +
  9. |   | Ax G(x)  .  A Intro: 4-8
     |   +
  10.| Ax G(x)      .  E Elim: 1, 3-9
\endfitchproof

Duplicate @-@s are optional, as are nearly all spaces; only spaces separating @||@ from a @-@ being used as negation are required (I include leading spaces for readibility, but they can also be omitted):

\nobreak
\endquantifiers
\fitchproof
 1.|AvB
 2.|A>C
 3.|B>D..Goal:-D>C
   |-
   |+
 4.|| -D..Assumption.Setting up:>Intro.Goal:C
   ||-
   ||+
 5.|||A..Assumption.Setting up:vElim.Goal:C
   |||-
 6.|||C.>Elim:2,5
   ||+
   ||+
   ||+
 7.|||B..Assumption.Setting up:vElim.Goal:C
   |||-
 8.|||D.>Elim:3,7
 9.|||!.!Intro:4,8
10.|||C.!Elim:9
   ||+
11.||C.vElim:1,5-6,7-10
   |+
12.| -D>C.>Intro:4-11
\endfitchproof
\quantifiers

\noindent A few additional notes:
\unorderedlist
\li If you don't use a colon after the name of a rule in a citation, the line numbers won't typeset quite right. 
\li A capital @R@ that begins a rule name (for {\bf Reit}) will be correctly typeset without math mode.
\li The capital @S@ is necessary for the {\bf Setting up} note to work properly.
\endunorderedlist

\pagebreak

\sec{Truth tables}

\unorderedlist
\li Truth tables automatically call @\endquantifiers@, so there's no need to use that before starting a truth table.
\li For correct spacing, every character should be in its own column, separated by periods.
\li In every row but the first, asterisks make the following character bold and larger. (If @\makeasteriskbig@ is active, @*@ continues to function as @\big@ in the first row.)
\li In every row but the first, brackets are ignored, so they can be included to make the code easier to read.
\li Only the first @+@ is required to create the horizontal line; subsequent @+@s are optional.
\endunorderedlist

The following examples illustrate the spacing and capitalization I personally use for formatting truth tables, all of which is optional.

\truthtable
 A | B | C : A. ^ .-.B | C. v .A | -.[.C. v .B.]
+++++++++++++++++++++++++++++++++++++++++++++++++
 T | T | T : t.*F .F.t | t.*T .t |*F.[.t. T .t.]
 T | T | F : t.*F .F.t | f.*T .t |*F.[.f. T .t.]
 T | F | T : t.*T .T.f | t.*T .t |*F.[.t. T .f.]
 T | F | F : t.*T .T.f | f.*T .t |*T.[.f. F .f.]
 F | T | T : f.*F .F.t | t.*T .f |*F.[.t. T .t.]
 F | T | F : f.*F .F.t | f.*F .f |*F.[.f. T .t.]
 F | F | T : f.*F .T.f | t.*T .f |*F.[.t. T .f.]
 F | F | F : f.*F .T.f | f.*F .f |*T.[.f. F .f.]
\endtruthtable

\noindent Again, all spacing and bar widths are customizable:

\truthtablerowheight=12.5pt \truthtablerowdepth=4.5pt%  Height/depth of each row
\truthtablefirstrowdepthadjust=2pt%  Depth increase for first row
\truthtablelinethickness=1.6pt%      Thickness of all lines
\truthtableinteritemspace=6pt%       Governs all horizontal spacing

\truthtable
 A | B | C : -.*[.A. v .[.B. ^ .C.]*]. v .[.A. v .B.]
+
 T | T | T : F. [.t. T .[.t. T .t.] ].*T .[.t. T .t.]
 T | T | F : F. [.t. T .[.t. F .f.] ].*T .[.t. T .t.]
 T | F | T : F. [.t. T .[.f. F .t.] ].*T .[.t. T .f.]
 T | F | F : F. [.t. T .[.f. F .f.] ].*T .[.t. T .f.]
 F | T | T : F. [.f. T .[.t. T .t.] ].*T .[.f. T .t.]
 F | T | F : T. [.f. F .[.t. F .f.] ].*T .[.f. T .t.]
 F | F | T : T. [.f. F .[.f. F .t.] ].*T .[.f. F .f.]
 F | F | F : T. [.f. F .[.f. F .f.] ].*T .[.f. F .f.]
\endtruthtable

\pagebreak

\endquantifiers
\sec{Arguments}

This is the argument the first truth table on the previous page proved invalid:

\argument
 A ^ -B
 C v A
++++++++++
 -[C v B]
\endargument

\noindent Changing spacing and the bar width:

\argumentpadding=6pt%        Padding between rows
\argumentsidepadding=6pt%    Padding on either side of each row
\argumentlinepadding=8pt%    Padding above and below the horizontal line
\argumentlinethickness=2pt%  Thickness of the horizontal line

\argument
 Taller(ruth,alonzo) v Taller(ruth,kurt)
 -[Shorter(kurt,alonzo) ^ Taller(ruth,kurt)]
+
 Taller(ruth,kurt) v Shorter(kurt,alonzo)
\endargument

\sec{Diagrams of relative height}

\unorderedlist
\li These are a simple way to give countermodels for validity of arguments about relative height.
\li Names must be in order from shortest to tallest.
\li When using these diagrams inline, they align with the words.
\endunorderedlist

\noindent Example:

\noindent The previous argument is invalid, as both\quad
\heightmodel
 Alonzo < Ruth = Kurt
\endheightmodel
\quad and\quad
\heightmodelinitialbarheight=8pt%  Initial height of the bars
\heightmodelbaradvance=4pt%        Amount to increase bar height
\heightmodelbarwidth=4pt%          Width of the bars
\heightmodelspace=4pt%             Horizontal space between items
\heightmodel
 A < R < K
\endheightmodel
\quad illustrate.

\bye
