% ---------------
% General utility
% ---------------
%
% These are macros that are used in more than one set of definitions below.

\def\ensureline{\hskip0pt}%  Make sure that the line has begun.
\def\logicstrut#1#2{\vrule height#1 depth#2 width0pt}
\def\makereturnsspaces{\catcode`\^^M=10\relax}
\def\ignoreallspaces{%
	\ifx\endignoreallspaces\undefined%
		\edef\endignoreallspaces{\catcode`\noexpand\ =\the\catcode`\ \relax}%
	\fi%
	\catcode`\ =9%
}


% -------------
% Abbreviations
% -------------
%
% \connectives and \quantifiers make several characters produce standard symbols
% used in logic. Specifically, they make occurrences of these characters in math
% mode behave like these expressions:
%
%			character	expression
%
%	\connectives:	v (lower-case)	\mathrel\lor
%			-		\lnot
%			^		\mathrel\land
%			>		\rightarrow
%			<		\leftrightarrow
%			!		\bot
%			=		\mathbin=
%
%	\quantifiers:	A		\forall
%			E		\exists
%
% In addition, \makeasteriskbig makes the * symbol equivalent to \big.
%
% I find these abbreviations helpful, but because the characters all have other
% uses meanings in TeX, these macros are likely to break other packages/macros.
% However, they are not required for any of the other macros in this document to
% function, so they can be freely omitted.
%
% I don't use ~ for negation or & for conjunction, so I've commented out the
% relevant lines, but I include those abbreviations, too, and all the macros are
% compatible with them (so they never use tilde as tie or ampersand as alignment
% tab).

\newif\ifconnectives \connectivesfalse
\def\connectives{%
	\ifconnectives%
	\else%
		\edef\endconnectives{%
			\noexpand\connectivesfalse%
			\mathcode`\noexpand\v=\the\mathcode`\v%
			\mathcode`\noexpand\==\the\mathcode`\=%
			\catcode`\noexpand\-=\the\catcode`\-%
%			\catcode`\noexpand\~=\the\catcode`\~%
			\catcode`\noexpand\^=\the\catcode`\^%
%			\catcode`\noexpand\&=\the\catcode`\&%
			\catcode`\noexpand\>=\the\catcode`\>%
			\catcode`\noexpand\<=\the\catcode`\<%
			\catcode`\noexpand\!=\the\catcode`\!%
			\mathcode`\noexpand\-=\the\mathcode`\-%
%			\mathcode`\noexpand\~=\the\mathcode`\~%
			\mathcode`\noexpand\^=\the\mathcode`\^%
%			\mathcode`\noexpand\&=\the\mathcode`\&%
			\mathcode`\noexpand\>=\the\mathcode`\>%
			\mathcode`\noexpand\<=\the\mathcode`\<%
			\mathcode`\noexpand\!=\the\mathcode`\!\relax%
		}%
	\fi%
	\connectivestrue%
	\mathcode`\v="325F%  \lor, but mathrel instead of mathbin
	\mathcode`\=="203D%  still =, but mathbin instead of mathrel
	\catcode`\-=12%
%	\catcode`\~=12%
	\catcode`\^=12%
%	\catcode`\&=12%
	\catcode`\>=12%
	\catcode`\<=12%
	\catcode`\!=12%
	\mathcode`\-="023A%  \lnot
%	\mathcode`\~="0218%  \sim, but mathord instead of mathrel
	\mathcode`\^="325E%  \land, but mathrel instead of mathbin
%	\mathcode`\&="3026%  still \&, but mathrel instead of mathord
	\mathcode`\>="3221%  \rightarrow
%	\mathcode`\>="321B%  \supset
	\mathcode`\<="3224%  \leftrightarrow
%	\mathcode`\<="3211%  \equiv
	\mathcode`\!="023F%  \bot
}

\newif\ifquantifiers \quantifiersfalse
\def\quantifiers{%
	\ifquantifiers%
	\else%
		\edef\endquantifiers{%
			\noexpand\quantifiersfalse%
			\mathcode`\noexpand\A=\the\mathcode`\A%
			\mathcode`\noexpand\E=\the\mathcode`\E\relax%
		}%
	\fi%
	\quantifierstrue%
	\mathcode`\A="0238%
	\mathcode`\E="0239%
}

\newif\ifasterisk \asteriskfalse
\begingroup
\catcode`\*=\active
\gdef\makeasteriskbig{%
	\ifasterisk%
	\else%
		\edef\resetasterisk{%
			\noexpand\asteriskfalse%
			\catcode`\noexpand\*=\the\catcode`\*\relax%
		}%
	\fi%
	\asterisktrue%
	\catcode`\*=\active%
	\let*=\big%
}
\endgroup


% ------------------
% Fitch-style proofs
% ------------------
%
% Notes:
%
% * Although spaces can usually be omitted in TeX, there must be a space
%   somewhere between a | and a subsequent - occurring as a negation.
%
% * Within \fitchproof, | makes spaces active, so don't use spaces in macro
%   definitions that could appear within \fitchproof.
%
% * Because . is used as alignment tab, it cannot be used in any other way. This
%   is why all dimens are named and defined in advance.
%
% * To ensure compatibility with \connectives, - also cannot be used.


% ----------------------------------
% Fitch-style proofs: Default values
% ----------------------------------

\newdimen\fitchrowheight    \newdimen\fitchrowdepth
\newdimen\fitchbarrowheight \newdimen\fitchbarrowdepth
\newdimen\fitchbarlength
\newdimen\fitchlinethickness
\newdimen\fitchpaddingthickness
\newdimen\fitchspace

\fitchrowheight=13.5pt \fitchrowdepth=7.5pt%   Rows containing formulas
\fitchbarrowheight=4pt \fitchbarrowdepth=6pt%  Rows containing horizontal bars
\fitchbarlength=48pt%                          Length of horizontal bars
\fitchlinethickness=1.2pt%                     Thickness of all lines
\fitchpaddingthickness=4pt%                    Padding produced by +
\fitchspace=6pt%                               Governs all horizontal spacing

% The widest connective is used to line up all "Intro"s and "Elim"s
\let\widestconnective=\rightarrow

% Three options for how to indent notes in a proof
\def\fitchnotesindent{\def\fitchnotesskip{\fitchskip}}
\def\fitchnotesnoindent{\def\fitchnotesskip{\relax}}
\def\fitchnotesnegindent{\def\fitchnotesskip{\fitchnegskip}}
\fitchnotesindent


% -------------------------------------------
% Fitch-style proofs: Preliminary definitions
% -------------------------------------------

\def\fitchstrut{\relax\logicstrut\fitchrowheight\fitchrowdepth}
\def\fitchbarstrut{\relax\logicstrut\fitchbarrowheight\fitchbarrowdepth}
\def\fitchpaddingstrut{\relax\logicstrut\fitchpaddingthickness{0pt}}

\newdimen\fitchspaceneg
\def\setfitchspaceneg{\fitchspaceneg=-\fitchspace}

\newdimen\fitchsetupskipamount
\def\setfitchsetupskip{%
	\setbox0\hbox{{\bf Setting up:{\rm\enspace}\spacedconnective\widestconnective \fitchsetuprule}}%
	\fitchsetupskipamount=\wd0%
	\def\fitchsetupskip{\hskip\fitchsetupskipamount{\rm\hskip.25em}\fitchskip}%
}

\def\resetfitchsetup{\def\fitchsetuprule{Intro}}
\def\widerfitchsetup{\def\fitchsetuprule{Elim:\rm\enspace0}}
\resetfitchsetup

\def\setfitchskips{%
	\setfitchspaceneg%
	\def\fitchskip{\hskip4\fitchspace}%
	\def\fitchnegskip{\hskip4\fitchspaceneg}%
	\setfitchsetupskip%
}

\def\fitchbar{\vtop{\hrule height\fitchlinethickness depth0pt width\fitchbarlength}}
\def\fitchline{\hskip\fitchspace\vrule width\fitchlinethickness\hskip2\fitchspace\makehyphenfitchbar}

\newif\ifisfitchnotes \isfitchnotesfalse
\begingroup
\catcode`\-=\active
\gdef\makehyphenendash{%
	\catcode`\-=\active%
	\def-{\char"7B\relax}%
}
\endgroup

\newdimen\connectivemaxwidth
\def\setconnectivemaxwidth{%
	\setbox0\hbox{$\widestconnective$}%
	\connectivemaxwidth=\wd0%
}

% The following three definitions ensure that \spacedconnective does nothing if
% it's followed by a macro whose expansion begins with \relax.

\def\spacedconnective{\afterassignment\spacedconnectiveA\let\next=}
\def\spacedconnectiveA{\expandafter\spacedconnectiveB\next}
\def\spacedconnectiveB#1{%
	\ifx#1\relax\else%
		\relax\setconnectivemaxwidth%
		\hbox to\connectivemaxwidth{\hss$#1$\hss}%
		\space\ignorespaces%
	\fi%
}

\begingroup
\catcode`\ =\active
\catcode`\-=\active
\catcode`\+=\active
\gdef\makehyphenfitchbar{%
	\ifx\resethyphen\undefined%
		\edef\resethyphen{\global\catcode`\noexpand\-=\the\catcode`\-\relax}%
	\fi%
	\catcode`\ =\active%
	\catcode`\-=\active%  TeX doesn't like \+ here, so use `+ instead of `\+
	\catcode`+=\active%   (I think the problem is that \+ is \outer).
	\let =\resethyphen%  Space, which is active, is now \resethyphen.
	\def-{$%
		\let-=\relax%
		\hskip2\fitchspaceneg%
		\fitchbar%
		\let\fitchstrut=\fitchbarstrut%
	$}%
	\def+{$%
		\let+=\relax%
		\let\fitchstrut\fitchpaddingstrut%
	$}%
}
\endgroup

\newdimen\fitchnamepadding \fitchnamepadding=3pt
\newdimen\fitchnamevadjust \fitchnamevadjust=-3.4pt
\begingroup
\catcode`\:=\active
\gdef\makecolonrm{%
	\catcode`\:\active%
	\def:{%
		\ifx\resetcapitalr\undefined \else\resetcapitalr \fi%
		\ifx\resetcapitals\undefined \else\resetcapitals \fi%
		\fitchnotesindent%
		\char"3A\rm%
		\ifisfitchnotes$ \else\makehyphenendash \fi%
		\enspace\ignorespaces%
	}%
}
\gdef\makecolonspacedconnective{%
	\catcode`\:\active%
	\def:{%
		\ignoreallspaces%
		\char"3A{\rm\enspace}%
		\makecolonrm%
		\spacedconnective%
	}%
}
\gdef\makecolonfitchname{%  This is slightly adapted from \boxit in the TeXbook
	\catcode`\:\active% via eplain.
	\def:{%
		\hskip\fitchnamepadding%
		\raise\fitchnamevadjust\vbox\bgroup%
			\hrule%
			\hbox\bgroup%
				\vrule%
				\kern\fitchnamepadding%
				\vbox\bgroup%
					\kern\fitchnamepadding\parindent=0pt%
					\hbox\bgroup$%
		\def:{%
					$\egroup\kern\fitchnamepadding%
				\egroup%
				\kern\fitchnamepadding%
				\vrule%
			\egroup%
			\hrule%
		\egroup\hskip2\fitchspace%
		}%
	}%
}
\endgroup

\newif\ifperiodprintsperiod
\begingroup
\catcode`\.=\active
\catcode`\S=\active
\gdef\makeperiodprintperiod{%
	\periodprintsperiodtrue%
	\catcode`\.=\active%
	\def.{\char"2E&\relax}%
}
\gdef\makeperiodspacedconnective{%
	\catcode`\.=\active%
	\def.{%
		\ifmmode$\fi&\relax%
		\spacedconnective%
	}%
}
\gdef\makeperiodnotes{%
	\catcode`\.=\active%
	\def.{%
		\relax%  Passed to \spacedconnectiveB.
		\fitchnotesskip%
		\isfitchnotestrue\makecapitalssetup%
		\def.{\fitchskip{\rm\enspace}}%
		\ignoreallspaces%
	}%
}
\gdef\makecapitalssetup{%
	\ifx\resetcapitals\undefined%
		\edef\resetcapitals{\catcode`\noexpand\S=\the\catcode`\S\relax}%
	\fi%
	\catcode`\.=\active%
	\catcode`\S=\active%
	\defS{%  \isfitchnotestrue will be reset by the next period
		\ifx\endignoreallspaces\undefined \else\endignoreallspaces \fi%
		\isfitchnotesfalse%  With the \hss in the definition of period
		\hbox to0pt\bgroup%  below, this \hbox is a manual \rlap.
			\char"53%  \char"50 = capital S
			\makecolonspacedconnective%
			\def.{\hss\egroup\resetcapitals\fitchsetupskip\isfitchnotestrue\ignorespaces}%
	}%
}
\endgroup

\begingroup
\catcode`\|=\active
\gdef\makepipefitchline{%
	\catcode`\|=\active%
	\def|{$\fitchline$}%
}
\gdef\makepipefitchtab{%
	\catcode`\|=\active%
	\def|{&$\fitchline$}%
}
\endgroup

\begingroup
\catcode`\R=\active
\gdef\makecapitalrreit{%
	\ifx\resetcapitalr\undefined%
		\edef\resetcapitalr{\catcode`\noexpand\R=\the\catcode`\R\relax}%
	\fi%
	\catcode`\R=\active%
	\defR{%
		\relax%  Passed to \spacedconnectiveB.
		\char"52%  \char"52 = capital R.
	}%
}
\endgroup


% -------------------------------------------------------------
% Fitch-style proofs: The \fitchproof and \endfitchproof macros
% -------------------------------------------------------------

\newtoks\everyfitchproof \everyfitchproof{\relax}
\newtoks\everyendfitchproof \everyendfitchproof{\relax}

\begingroup
\catcode`\.=\active%  Period will be used as alignment tab.
\gdef\fitchproofcore#1{\ensureline\vtop\bgroup\the\everyfitchproof
	#1
	\setfitchskips
	\periodprintsperiodfalse
	\catcode`\.=\active
	\offinterlineskip
	\def\par{\relax\ifmmode$\fi\ifperiodprintsperiod\else.\fi\cr\ignorespaces}
	\obeylines%
	\halign\bgroup%
		\makepipefitchtab%            Column 1: line numbers
		\makeperiodprintperiod%
		\hfil##\hskip\fitchspace&%
		\makepipefitchline%           Column 2: indentation and formulas
		\makeperiodspacedconnective%
		\makecolonfitchname%
		$##\fitchstrut\hfil&%
		\makeperiodnotes%             Column 3: citations and notes
		\makecolonrm%
		\makecapitalrreit%
		\isfitchnotesfalse%
		\fitchskip\bf##\hfil\cr%
		\makereturnsspaces
}%  The previous line ensures that blank lines immediately following \fitchproof
\endgroup%  are ignored. It does not end with %, so TeX sees a space.

\def\fitchproof{\fitchproofcore{}}
\def\fitchproofindent{\fitchproofcore{\fitchnotesindent}}
\def\fitchproofnoindent{\fitchproofcore{\fitchnotesnoindent}}
\def\fitchproofnegindent{\fitchproofcore{\fitchnotesnegindent}}

\def\endfitchproof{\cr\egroup\egroup\the\everyendfitchproof}


% ------------
% Truth tables
% ------------
%
% Notes:
%
% * Because . is used as alignment tab, it cannot be used in any other way. This
%   is why all dimens are named and defined in advance.
%
% * To ensure compatibility with \connectives, - also cannot be used.

% ----------------------------
% Truth tables: Default values
% ----------------------------

\newdimen\truthtablerowheight \newdimen\truthtablerowdepth
\newdimen\truthtablefirstrowdepthadjust
\newdimen\truthtablelinethickness
\newdimen\truthtableinteritemspace

\truthtablerowheight=10.5pt \truthtablerowdepth=3.5pt%  Height/depth of each row
\truthtablefirstrowdepthadjust=1pt%  Depth increase for first row
\truthtablelinethickness=1.2pt%      Thickness of all lines
\truthtableinteritemspace=4pt%       Governs all horizontal spacing


% -------------------------------------
% Truth tables: Preliminary definitions
% -------------------------------------

\font\eightrm=cmr8
\font\elevenbf=cmbx10 scaled \magstephalf

\def\ttstrut{\relax\logicstrut\truthtablerowheight\truthtablerowdepth}
\def\ttfirstlinestrut{{\advance\truthtablerowdepth by\truthtablefirstrowdepthadjust\ttstrut}}

\newdimen\tthalfinteritemspace
\def\settthalfinteritemspace{\tthalfinteritemspace=.5\truthtableinteritemspace}

\newif\ifttfirstrow
\def\ttvrule{{%
	\vrule width\truthtablelinethickness%
	\ifttfirstrow\ttfirstlinestrut\else\ttstrut\fi%
}}

\def\makettcellmath{\def\ttcell{${}}}
\def\makettcellvalues{\global\def\ttcell{\eightrm\makeasteriskttValue}}

\newdimen\ttValueshift \ttValueshift=-.7pt
\let\ttValuefont=\elevenbf

% The following three definitions ensure that \ttValue does nothing if it's
% followed by a macro whose expansion begins with \relax.

\def\ttValue{\afterassignment\ttValueA\let\next=}
\def\ttValueA{\expandafter\ttValueB\next}
\def\ttValueB#1{%
	\ifx#1\relax\else%
		\raise\ttValueshift\hbox{\ttValuefont#1}%
	\fi%
}

\begingroup
\catcode`\*=\active
\gdef\makeasteriskttValue{%
	\catcode`\*=\active%
	\let*\ttValue\relax%
}
\endgroup

\def\truthtableline{%
	\ifmmode$\fi\cr\noalign{%
		\hrule height\truthtablelinethickness%
		\global\ttfirstrowfalse%
		\makettcellvalues%
	}%
}


% -------------------------------------------------------
% Truth tables: The \truthtable and \endtruthtable macros
% -------------------------------------------------------

\newtoks\everytruthtable \everytruthtable{\relax}
\newtoks\everyendtruthtable \everyendtruthtable{\relax}

\begingroup
\catcode`\.=\active%  Period will be used as alignment tab.
\catcode`\|=\active
\catcode`\:=\active
\catcode`\T=\active
\catcode`\F=\active
\gdef\truthtable{\ensureline\vtop\bgroup\the\everytruthtable
	\ifx\endquantifiers\undefined \else\endquantifiers \fi%  A and E work
	\catcode`\.=\active%                                     normally in
	\catcode`\|=\active%                                     truth tables.
	\catcode`\:=\active
	\def\endcell{{}\ifmmode$\fi&}
	\def.{\relax\endcell\relax}%  The initial \relax is to be passed to *.
	\def|{.\endcell\omit\ttvrule..}
	\def:{.\endcell\omit\ttvrule\thinspace\ttvrule..}
	\offinterlineskip
	\ttfirstrowtrue
	\tabskip=\truthtableinteritemspace
	\settthalfinteritemspace
	\makettcellmath
	\ignoreallspaces
	\def\par{\endcell\omit\hskip-\tthalfinteritemspace\cr{}}
	\obeylines%
	\halign\bgroup%
		\hskip\tthalfinteritemspace\hfil\ttcell##\hfil&&%
		\hfil\ttcell##{}\hfil\ttstrut\cr%
		\makereturnsspaces
}%  The previous line ensures that blank lines immediately following \truthtable
\endgroup%  are ignored. It does not end with %, so TeX sees a space.

\def\endtruthtable{\crcr\egroup\egroup\the\everyendtruthtable}


% ------------------
% First-order models
% ------------------

\newdimen\fomodelfirstrowheight \newdimen\fomodelfirstrowdepth
\newdimen\fomodelrowheight      \newdimen\fomodelrowdepth
\newdimen\fomodelpadding
\newdimen\fomodelspace

\fomodelfirstrowheight=8.5pt \fomodelfirstrowdepth=8.5pt%  First row (domain)
\fomodelrowheight=11pt       \fomodelrowdepth=3.5pt%       Subsequent rows
\fomodelpadding=1.5pt%  Padding between rows
\fomodelspace=12pt%     Horizontal space after colons


\newif\iffomodelfirstrow
\def\fomodelstrut{%
	\iffomodelfirstrow\logicstrut{\fomodelfirstrowheight}{\fomodelfirstrowdepth}%
	\else\logicstrut{\fomodelrowheight}{\fomodelrowdepth}%
	\fi%
}

\begingroup
\catcode`\<=\active
\catcode`\>=\active
\gdef\makefomodelangles{%
	\catcode`\<=\active%
	\catcode`\>=\active%
	\def<{$\langle$\resetcommasf}%
	\def>{$\rangle$\increasecommasf}%
}
\endgroup

\def\increasecommasf{\sfcode`\,2000}

\begingroup
\catcode`\:=\active%  Colon will be used as alignment tab.
\catcode`\.=\active%  Period will also sometimes be used as alignment tab.
\gdef\firstordermodel{\ensureline\vtop\bgroup%
	\catcode`\:=\active
	\catcode`\.=\active
	\edef\resetcommasf{\sfcode`\noexpand\,=\the\sfcode`\,\relax}
	\increasecommasf
	\fomodelfirstrowtrue
	\offinterlineskip
	\openup\fomodelpadding
	\def\fomodelalignmenttab##1{%
		\ifmmode$\fi%
		\rlap{##1}%
		\global\fomodelfirstrowfalse%
		\hskip\fomodelspace%
		&\ignorespaces%
	}
	\def:{\fomodelstrut\fomodelalignmenttab{\char"3A}}
	\def.{\strut\fomodelalignmenttab{\relax}}
	\def\par{\ifmmode$\fi\cr\ignorespaces}
	\obeylines%
	\halign\bgroup%
		\iffomodelfirstrow\bf\else$\fi%
		\hfil##&%
		\vtop{\normalbaselines\makefomodelangles##}\hfil\cr%
		\makereturnsspaces
}%          The previous line ensures that blank lines immediately following
\endgroup%  \firstordermodel are ignored. It does not end with %, so TeX sees a
%           space.

\def\endfirstordermodel{{}\ifmmode$\fi\cr\egroup\egroup}


% ---------
% Arguments
% ---------

\newdimen\argumentpadding
\newdimen\argumentsidepadding
\newdimen\argumentlinepadding
\newdimen\argumentlinethickness

\argumentpadding=3pt%          Padding between rows (3pt = \jot)
\argumentsidepadding=3pt%      Padding on either side of each row
\argumentlinepadding=5pt%      Padding above and below the horizontal line
\argumentlinethickness=1.2pt%  Thickness of the horizontal line


\def\argumentline{\crcr\noalign{\kern\argumentlinepadding\hrule height\argumentlinethickness\kern\argumentlinepadding}}

\def\argument{\ensureline\vtop\bgroup
	\openup\argumentpadding
	\def\par{\crcr}
	\halign\bgroup%
		\obeylines%
		\hskip\argumentsidepadding%
		$##$%
		\hskip\argumentsidepadding%
		\hfil\cr%
}
\def\endargument{\crcr\egroup\egroup}


% ----------------------
% Alternate math letters
% ----------------------

\def\setfam#1#2#3#4#5{%
	\textfont#1=\csname#3#2\endcsname%
	\scriptfont#1=\csname#4#2\endcsname%
	\scriptscriptfont#1=\csname#5#2\endcsname%
}

\def\makenumeralsfam#1{%
	\mathcode`\0="0#130%
	\mathcode`\1="0#131%
	\mathcode`\2="0#132%
	\mathcode`\3="0#133%
	\mathcode`\4="0#134%
	\mathcode`\5="0#135%
	\mathcode`\6="0#136%
	\mathcode`\7="0#137%
	\mathcode`\8="0#138%
	\mathcode`\9="0#139%
}

\font\tenss=cmss10
\font\sevenss=cmss10 scaled 700%  Using cmss10 scaled because smaller cmss fonts
\font\fivess=cmss10 scaled 500%   look ugly.

\def\ssmath{%
	\setfam{15}{ss}{ten}{seven}{five}%
	\everymath{\fam15}%
	\makenumeralsfam0%
}


% --------------------------------------------
% Italic math letters with normal text spacing
% --------------------------------------------

\font\sevenit=cmti7
\font\fiveit=cmti5

\def\itmath{%
	\setfam{15}{it}{ten}{seven}{five}%
	\everymath{\fam15}%
	\makenumeralsfam0%
}


% -------------------------
% Models of relative height
% -------------------------
%
% These are a simple way to demonstrate invalidity for arguments about relative 
% heights/sizes.

\newdimen\heightmodelinitialbarheight
\newdimen\heightmodelbaradvance
\newdimen\heightmodelbarwidth
\newdimen\heightmodelspace

\heightmodelinitialbarheight=16pt%  Initial height of the bars
\heightmodelbaradvance=8pt%         Amount to increase bar height
\heightmodelbarwidth=6pt%           Width of the bars
\heightmodelspace=8pt%              Horizontal space between items


\newdimen\heightmodelbarheight
\def\resetheightmodelbarheight{\heightmodelbarheight=\heightmodelinitialbarheight}

\def\heightmodelbar{\vrule height\heightmodelbarheight width\heightmodelbarwidth}

\def\heightmodelitem{%
	\ensureline\vbox\bgroup%
		\halign\bgroup%
			\hss##\hss\cr%
			\hbox{\heightmodelbar}\cr%
}
\def\endheightmodelitem{\cr\egroup\egroup}

\begingroup
\catcode`\<=\active
\catcode`\==\active%  Don't use = after this.
\gdef\heightmodel{\ensureline\hbox\bgroup%
	\resetheightmodelbarheight%
	\catcode`\<\active%
	\catcode`\=\active%
	\def={%
		\endheightmodelitem%
		\hskip\heightmodelspace%
		\heightmodelitem%
	}%
	\def<{%
		\endheightmodelitem%
		\advance\heightmodelbarheight by\heightmodelbaradvance%
		\hskip\heightmodelspace%
		\heightmodelitem%
	}%
	\ignoreallspaces%
	\let\par\relax%
	\obeylines%
	\heightmodelitem%
}
\endgroup

\def\endheightmodel{\endheightmodelitem\egroup}
